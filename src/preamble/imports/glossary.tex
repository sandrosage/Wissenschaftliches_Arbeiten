% Robin Prillwitz 2022

\usepackage[
  acronym, toc
]{glossaries}
\usepackage[automake, nonumberlist, nogroupskip]{glossaries-extra}

\renewcommand*{\glsclearpage}{} % remove gloassary pagebreak
% \renewcommand*{\@@glossarysec}{section}
\setabbreviationstyle[acronym]{long-short}
\renewcommand*{\glstextformat}[1]{\textit{#1}} % cursive acronyms
\renewcommand\glstreepredesc{\tabto{4cm}} % glossary spacing

% --------------------------------- Acronyms -------------------------------- %
\newacronym{ml}{ML}{Machine Learning}
\newacronym{ki}{KI}{Künstliche Intelligenz}
\newacronym[]{cart}{CART}{Classification and regression trees}
\newacronym[]{knn}{KNN}{K-nearest neighbors}
\newacronym[]{cv}{CV}{Cross-Validation}

% --------------------------------- Glossary -------------------------------- %
\newglossaryentry{bl}{name={BLDC-Motor},description={Bürstenloser Gleichstrommotor},plural={BLDC-Motoren}}
\newglossaryentry{weaklearner}{name={Weak Lerner},description={Schwacher Lerner - Lernalgorithmus der mindestens eine Verallgemeinerungsrate von über 0.5 besitzt}}
\newglossaryentry{label}{name={Label},description={(Klassen)-Bezeichnung - gebräuchlicher englischer Ausdruck}}
\newglossaryentry{sample}{name={Sample},description={Datenpunkt - gebräuchlicher englischer Ausdruck}}
\newglossaryentry{baselearner}{name={Base Learner},description={Basislerner - individuelle Lernalgorithmen in homogenen Ensembles}}
\newglossaryentry{output}{name={Output},description={Ausgabe - herkömmlicher englischer Begriff}}
\makeglossaries
